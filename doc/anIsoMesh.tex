\documentclass{article}

\title{Anisotropic Meshing Metrics} 

\author{Michael A. Park
  \thanks{Research Scientist, Computational Modeling and Simulation Branch,
          \texttt{m.a.park@larc.nasa.gov}} \\
  {\itshape NASA Langley Research Center,~Hampton,~VA~23681}\\[5pt]
}

\begin{document}

\maketitle 

\section{Transformation Matrix}

The Jacobian $\mathbf{J}$ is employed to map physical space
$\mathbf{x}=\left[x ~y~z\right]^t$ to a transformed space
$\mathbf{e}=\left[e~f~g\right]^t$ where triangle or tetrahedron
becomes equilateral.
\begin{equation}\label{e:x}
 \mathbf{e}=\mathbf{J}\mathbf{x}
\end{equation}
\begin{equation}
 \mathbf{J}=\left[
 \begin{array}{ccc}
   \frac{\partial e}{\partial x}&
   \frac{\partial e}{\partial y}&
   \frac{\partial e}{\partial z}\\
   \frac{\partial f}{\partial x}&
   \frac{\partial f}{\partial y}&
   \frac{\partial f}{\partial z}\\
   \frac{\partial g}{\partial x}&
   \frac{\partial g}{\partial y}&
   \frac{\partial g}{\partial z}
 \end{array}\right]
\end{equation}
The $\mathbf{J}$ matrix is not required to be symmetric.

The Metric Tensor $\mathbf{M}$ is defined as 
\begin{equation}\label{metric:jacobian}
 \mathbf{M}=\mathbf{J}^t\mathbf{J}
\end{equation}
\begin{equation}
 \mathbf{M}=\left[
 \begin{array}{ccc}
   \frac{\partial^2 e}{\partial x^2}&
   \frac{\partial e \partial f}{\partial y \partial x}&
   \frac{\partial e \partial g}{\partial z \partial x}\\
   \frac{\partial f \partial e}{\partial x \partial y}&
   \frac{\partial^2 f}{\partial y^2}&
   \frac{\partial f \partial g}{\partial z\partial y}\\
   \frac{\partial g \partial e}{\partial x\partial z}&
   \frac{\partial g \partial f}{\partial y\partial z}&
   \frac{\partial^2 g}{\partial z^2}
 \end{array}\right]
\end{equation}
The matrix $\mathbf{M}$ is symmetric and can be decomposed into Eigen
values $\Lambda$ and Eigen vectors $\mathbf{R}$
\begin{equation}\label{metric:eigen}
 \mathbf{M}=\mathbf{R}\Lambda\mathbf{R}^t
\end{equation}
where the spacing $h_i$ is given as 
\begin{equation}
 \Lambda=\left[
 \begin{array}{ccc}
   \frac{1}{h_1^2}&&\\
   &\frac{1}{h_2^2}&\\
   &&\frac{1}{h_3^2}\\
 \end{array}\right]
\end{equation}
The $\frac{1}{h^2}$ can be shown by combing Equ.~\ref{metric:jacobian}
and Equ.~\ref{metric:eigen}
\begin{equation}
 \mathbf{M}=\mathbf{R}\Lambda^\frac{1}{2}\Lambda^\frac{1}{2}\mathbf{R}^t
\end{equation}
\begin{equation}
 \mathbf{J}=\Lambda^\frac{1}{2}\mathbf{R}^t
\end{equation}

\section{Transformed Length}

The length $l$ in mapped space is
\begin{equation}
 l=\sqrt{\mathbf{e}^t\mathbf{e}}
\end{equation}
Employing Equ.~\ref{e:x} this expression becomes
\begin{equation}
 l=\sqrt{ \left( \mathbf{J}\mathbf{x}\right)^t \left( \mathbf{J}\mathbf{x} \right) }
\end{equation}
\begin{equation}
 l=\sqrt{ \mathbf{x}^t\mathbf{J}^t\mathbf{J}\mathbf{x}}
\end{equation}
\begin{equation}
 l=\sqrt{ \mathbf{x}^t\mathbf{M}\mathbf{x}}
\end{equation}

\section{Pure Scaling Transformation Example}

Map a sphere of diameter 2 to a unit sphere. 
The requested spacing is therefore $h=2$.
Which is the following Eigen values
\begin{equation}
 \Lambda^\frac{1}{2}=\left[
 \begin{array}{ccc}
   \frac{1}{2}&&\\
   &\frac{1}{2}&\\
   &&\frac{1}{2}\\
 \end{array}\right]
\end{equation}
\begin{equation}
 \Lambda=\left[
 \begin{array}{ccc}
   \frac{1}{4}&&\\
   &\frac{1}{4}&\\
   &&\frac{1}{4}\\
 \end{array}\right]
\end{equation}
and Eigen Vectors
\begin{equation}
 \mathbf{R}=\left[
 \begin{array}{ccc}
   1&&\\
   &1&\\
   &&1\\
 \end{array}\right]
\end{equation}
The Jacobian of transformation is
\begin{equation}
 \mathbf{J}=\Lambda^\frac{1}{2}\mathbf{R}^t=\left[
 \begin{array}{ccc}
   \frac{1}{2}&&\\
   &\frac{1}{2}&\\
   &&\frac{1}{2}\\
 \end{array}\right]
\end{equation}
The Metric Matrix is 
\begin{equation}
 \mathbf{M}=
  \mathbf{J}^t\mathbf{J}=
  \mathbf{R}\Lambda^\frac{1}{2}\Lambda^\frac{1}{2}\mathbf{R}^t=
  \mathbf{R}\Lambda\mathbf{R}^t=
  \left[\begin{array}{ccc}
   \frac{1}{4}&&\\
   &\frac{1}{4}&\\
   &&\frac{1}{4}\\
 \end{array}\right]
\end{equation}

Where a radius is given as
\begin{equation}
 \mathbf{r}=\left[2~0~0\right]^t
\end{equation}
This vector is mapped to the transformed space
\begin{equation}
 \mathbf{J}\mathbf{r}=\left[
 \begin{array}{ccc}
   \frac{1}{2}&&\\
   &\frac{1}{2}&\\
   &&\frac{1}{2}\\
 \end{array}\right]\left[2~0~0\right]^t=\left[1~0~0\right]^t
\end{equation}
And its length is
\begin{equation}
 r=\sqrt{\mathbf{r}^t\mathbf{M}\mathbf{r}}
\end{equation}
\begin{equation}
 r=\sqrt{\left[2~0~0\right]\left[
 \begin{array}{ccc}
   \frac{1}{4}&&\\
   &\frac{1}{4}&\\
   &&\frac{1}{4}\\
 \end{array}\right]\left[2~0~0\right]^t}=1
\end{equation}

\section{Rotaion and Scaling Transformation Example}

Map three vectors to unit length.

\begin{equation}
 \mathbf{v}_1=\left[
 \begin{array}{c}
   2\\
   2\\
   0\\
 \end{array}\right]
\end{equation}
\begin{equation}
 \mathbf{v}_2=\left[
 \begin{array}{c}
   -1\\
   1\\
   0\\
 \end{array}\right]
\end{equation}
\begin{equation}
 \mathbf{v}_3=\left[
 \begin{array}{c}
   0\\
   0\\
   1\\
 \end{array}\right]
\end{equation}
 
The requested spacing is therefore $h_1=2\sqrt{2}$, $h_2=\sqrt{2}$, .$h_3=1$, 
Which is the following Eigen values
\begin{equation}
 \Lambda^\frac{1}{2}=
 \left[\begin{array}{ccc}
   \frac{1}{h_1}&&\\
   &\frac{1}{h_2}&\\
   &&\frac{1}{h_3}\\
 \end{array}\right]
 =
 \left[\begin{array}{ccc}
   \frac{1}{2\sqrt{2}}&&\\
   &\frac{1}{\sqrt{2}}&\\
   &&1\\
 \end{array}\right]
\end{equation}
\begin{equation}
 \Lambda=
 \left[\begin{array}{ccc}
   \frac{1}{h_1^2}&&\\
   &\frac{1}{h_2^2}&\\
   &&\frac{1}{h_3^2}\\
 \end{array}\right]
 =
 \left[\begin{array}{ccc}
   \frac{1}{8}&&\\
   &\frac{1}{2}&\\
   &&1\\
 \end{array}\right]
\end{equation}
and Eigen Vectors
\begin{equation}
 \mathbf{R}=\left[
 \begin{array}{ccc}
   \frac{1}{\sqrt{2}}&-\frac{1}{\sqrt{2}}&0\\
   \frac{1}{\sqrt{2}}&\frac{1}{\sqrt{2}}&0\\
   0&0&1\\
 \end{array}\right]
\end{equation}
The Jacobian of transformation is
\begin{equation}
 \mathbf{J}=\Lambda^\frac{1}{2}\mathbf{R}^t=
 \left[\begin{array}{ccc}
   \frac{1}{2\sqrt{2}}&&\\
   &\frac{1}{\sqrt{2}}&\\
   &&1\\
 \end{array}\right]
 \left[\begin{array}{ccc}
   \frac{1}{\sqrt{2}}&\frac{1}{\sqrt{2}}&0\\
   -\frac{1}{\sqrt{2}}&\frac{1}{\sqrt{2}}&0\\
   0&0&1\\
 \end{array}\right]
 =
 \left[\begin{array}{ccc}
   \frac{1}{4}&\frac{1}{4}&0\\
   -\frac{1}{2}&\frac{1}{2}&0\\
   0&0&1\\
 \end{array}\right]
\end{equation}
The Metric Matrix is 
\begin{equation}
 \mathbf{M}=
  \mathbf{J}^t\mathbf{J}=
  \mathbf{R}\Lambda^\frac{1}{2}\Lambda^\frac{1}{2}\mathbf{R}^t=
  \mathbf{R}\Lambda\mathbf{R}^t
\end{equation}
\begin{equation}
 \left[\begin{array}{ccc}
   \frac{1}{4}&-\frac{1}{2}&0\\
   \frac{1}{4}&\frac{1}{2}&0\\
   0&0&1\\
 \end{array}\right]
 \left[\begin{array}{ccc}
   \frac{1}{4}&\frac{1}{4}&0\\
   -\frac{1}{2}&\frac{1}{2}&0\\
   0&0&1\\
 \end{array}\right]
 =  
  \left[\begin{array}{ccc}
   \frac{5}{16}&-\frac{3}{16}&0\\
   -\frac{3}{16}&\frac{5}{16}&0\\
   0&0&1\\
 \end{array}\right]
\end{equation}
The tranformed vectors are
\begin{equation}
 \left[
 \begin{array}{ccc}
   \frac{1}{4}&\frac{1}{4}&0\\
   -\frac{1}{2}&\frac{1}{2}&0\\
   0&0&1\\
 \end{array}\right]
 \left[
 \begin{array}{c}
   2\\
   2\\
   0\\
 \end{array}\right]
 =\left[
 \begin{array}{c}
   1\\
   0\\
   0\\
 \end{array}\right]
\end{equation}
\begin{equation}
 \left[
 \begin{array}{ccc}
   \frac{1}{4}&\frac{1}{4}&0\\
   -\frac{1}{2}&\frac{1}{2}&0\\
   0&0&1\\
 \end{array}\right]
 \left[
 \begin{array}{c}
   -1\\
   1\\
   0\\
 \end{array}\right]
 =\left[
 \begin{array}{c}
   0\\
   1\\
   0\\
 \end{array}\right]
\end{equation}
\begin{equation}
 \left[
 \begin{array}{ccc}
   \frac{1}{4}&\frac{1}{4}&0\\
   -\frac{1}{2}&\frac{1}{2}&0\\
   0&0&1\\
 \end{array}\right]
 \left[
 \begin{array}{c}
   0\\
   0\\
   1\\
 \end{array}\right]
 =\left[
 \begin{array}{c}
   0\\
   0\\
   1\\
 \end{array}\right]
\end{equation}

The lengths in the mapped space are
\begin{equation}
 \sqrt{
 \left[\begin{array}{ccc}
   2&2&0\\
 \end{array}\right] 
 \left[\begin{array}{ccc}
   \frac{5}{16}&-\frac{3}{16}&0\\
   -\frac{3}{16}&\frac{5}{16}&0\\
   0&0&1\\
 \end{array}\right]
 \left[\begin{array}{c}
   2\\
   2\\
   0\\
 \end{array}\right]
 }=1
\end{equation}
\begin{equation}
 \sqrt{
 \left[\begin{array}{ccc}
   -1&1&0\\
 \end{array}\right] 
 \left[\begin{array}{ccc}
   \frac{5}{16}&-\frac{3}{16}&0\\
   -\frac{3}{16}&\frac{5}{16}&0\\
   0&0&1\\
 \end{array}\right]
 \left[\begin{array}{c}
   -1\\
   1\\
   0\\
 \end{array}\right]
 }=1
\end{equation}
\begin{equation}
 \sqrt{
 \left[\begin{array}{ccc}
   0&0&1\\
 \end{array}\right] 
 \left[\begin{array}{ccc}
   \frac{5}{16}&-\frac{3}{16}&0\\
   -\frac{3}{16}&\frac{5}{16}&0\\
   0&0&1\\
 \end{array}\right]
 \left[\begin{array}{c}
   0\\
   0\\
   1\\
 \end{array}\right]
 }=1
\end{equation}


\end{document}